% Options for packages loaded elsewhere
\PassOptionsToPackage{unicode}{hyperref}
\PassOptionsToPackage{hyphens}{url}
%
\documentclass[
]{article}
\usepackage{lmodern}
\usepackage{amsmath}
\usepackage{ifxetex,ifluatex}
\ifnum 0\ifxetex 1\fi\ifluatex 1\fi=0 % if pdftex
  \usepackage[T1]{fontenc}
  \usepackage[utf8]{inputenc}
  \usepackage{textcomp} % provide euro and other symbols
  \usepackage{amssymb}
\else % if luatex or xetex
  \usepackage{unicode-math}
  \defaultfontfeatures{Scale=MatchLowercase}
  \defaultfontfeatures[\rmfamily]{Ligatures=TeX,Scale=1}
\fi
% Use upquote if available, for straight quotes in verbatim environments
\IfFileExists{upquote.sty}{\usepackage{upquote}}{}
\IfFileExists{microtype.sty}{% use microtype if available
  \usepackage[]{microtype}
  \UseMicrotypeSet[protrusion]{basicmath} % disable protrusion for tt fonts
}{}
\makeatletter
\@ifundefined{KOMAClassName}{% if non-KOMA class
  \IfFileExists{parskip.sty}{%
    \usepackage{parskip}
  }{% else
    \setlength{\parindent}{0pt}
    \setlength{\parskip}{6pt plus 2pt minus 1pt}}
}{% if KOMA class
  \KOMAoptions{parskip=half}}
\makeatother
\usepackage{xcolor}
\IfFileExists{xurl.sty}{\usepackage{xurl}}{} % add URL line breaks if available
\IfFileExists{bookmark.sty}{\usepackage{bookmark}}{\usepackage{hyperref}}
\hypersetup{
  pdftitle={R vs Python},
  pdfauthor={Claude REN},
  hidelinks,
  pdfcreator={LaTeX via pandoc}}
\urlstyle{same} % disable monospaced font for URLs
\usepackage[margin=1in]{geometry}
\usepackage{color}
\usepackage{fancyvrb}
\newcommand{\VerbBar}{|}
\newcommand{\VERB}{\Verb[commandchars=\\\{\}]}
\DefineVerbatimEnvironment{Highlighting}{Verbatim}{commandchars=\\\{\}}
% Add ',fontsize=\small' for more characters per line
\usepackage{framed}
\definecolor{shadecolor}{RGB}{248,248,248}
\newenvironment{Shaded}{\begin{snugshade}}{\end{snugshade}}
\newcommand{\AlertTok}[1]{\textcolor[rgb]{0.94,0.16,0.16}{#1}}
\newcommand{\AnnotationTok}[1]{\textcolor[rgb]{0.56,0.35,0.01}{\textbf{\textit{#1}}}}
\newcommand{\AttributeTok}[1]{\textcolor[rgb]{0.77,0.63,0.00}{#1}}
\newcommand{\BaseNTok}[1]{\textcolor[rgb]{0.00,0.00,0.81}{#1}}
\newcommand{\BuiltInTok}[1]{#1}
\newcommand{\CharTok}[1]{\textcolor[rgb]{0.31,0.60,0.02}{#1}}
\newcommand{\CommentTok}[1]{\textcolor[rgb]{0.56,0.35,0.01}{\textit{#1}}}
\newcommand{\CommentVarTok}[1]{\textcolor[rgb]{0.56,0.35,0.01}{\textbf{\textit{#1}}}}
\newcommand{\ConstantTok}[1]{\textcolor[rgb]{0.00,0.00,0.00}{#1}}
\newcommand{\ControlFlowTok}[1]{\textcolor[rgb]{0.13,0.29,0.53}{\textbf{#1}}}
\newcommand{\DataTypeTok}[1]{\textcolor[rgb]{0.13,0.29,0.53}{#1}}
\newcommand{\DecValTok}[1]{\textcolor[rgb]{0.00,0.00,0.81}{#1}}
\newcommand{\DocumentationTok}[1]{\textcolor[rgb]{0.56,0.35,0.01}{\textbf{\textit{#1}}}}
\newcommand{\ErrorTok}[1]{\textcolor[rgb]{0.64,0.00,0.00}{\textbf{#1}}}
\newcommand{\ExtensionTok}[1]{#1}
\newcommand{\FloatTok}[1]{\textcolor[rgb]{0.00,0.00,0.81}{#1}}
\newcommand{\FunctionTok}[1]{\textcolor[rgb]{0.00,0.00,0.00}{#1}}
\newcommand{\ImportTok}[1]{#1}
\newcommand{\InformationTok}[1]{\textcolor[rgb]{0.56,0.35,0.01}{\textbf{\textit{#1}}}}
\newcommand{\KeywordTok}[1]{\textcolor[rgb]{0.13,0.29,0.53}{\textbf{#1}}}
\newcommand{\NormalTok}[1]{#1}
\newcommand{\OperatorTok}[1]{\textcolor[rgb]{0.81,0.36,0.00}{\textbf{#1}}}
\newcommand{\OtherTok}[1]{\textcolor[rgb]{0.56,0.35,0.01}{#1}}
\newcommand{\PreprocessorTok}[1]{\textcolor[rgb]{0.56,0.35,0.01}{\textit{#1}}}
\newcommand{\RegionMarkerTok}[1]{#1}
\newcommand{\SpecialCharTok}[1]{\textcolor[rgb]{0.00,0.00,0.00}{#1}}
\newcommand{\SpecialStringTok}[1]{\textcolor[rgb]{0.31,0.60,0.02}{#1}}
\newcommand{\StringTok}[1]{\textcolor[rgb]{0.31,0.60,0.02}{#1}}
\newcommand{\VariableTok}[1]{\textcolor[rgb]{0.00,0.00,0.00}{#1}}
\newcommand{\VerbatimStringTok}[1]{\textcolor[rgb]{0.31,0.60,0.02}{#1}}
\newcommand{\WarningTok}[1]{\textcolor[rgb]{0.56,0.35,0.01}{\textbf{\textit{#1}}}}
\usepackage{graphicx}
\makeatletter
\def\maxwidth{\ifdim\Gin@nat@width>\linewidth\linewidth\else\Gin@nat@width\fi}
\def\maxheight{\ifdim\Gin@nat@height>\textheight\textheight\else\Gin@nat@height\fi}
\makeatother
% Scale images if necessary, so that they will not overflow the page
% margins by default, and it is still possible to overwrite the defaults
% using explicit options in \includegraphics[width, height, ...]{}
\setkeys{Gin}{width=\maxwidth,height=\maxheight,keepaspectratio}
% Set default figure placement to htbp
\makeatletter
\def\fps@figure{htbp}
\makeatother
\setlength{\emergencystretch}{3em} % prevent overfull lines
\providecommand{\tightlist}{%
  \setlength{\itemsep}{0pt}\setlength{\parskip}{0pt}}
\setcounter{secnumdepth}{-\maxdimen} % remove section numbering
\newenvironment{knitrout}{\setlength{\topsep}{0mm}}{}
%% See: https://bookdown.org/yihui/rmarkdown-cookbook/multi-column-layout.html
%% grantmcdermott GitHub/two-col-test
\newenvironment{columns}[1][]{}{}
%%
\newenvironment{column}[1]{\begin{minipage}[t]{#1}\ignorespaces}{%
\end{minipage}
\ifhmode\unskip\fi
\aftergroup\useignorespacesandallpars}
%%
\def\useignorespacesandallpars#1\ignorespaces\fi{%
#1\fi\ignorespacesandallpars}
%%
\makeatletter
\def\ignorespacesandallpars{%
  \@ifnextchar\par
    {\expandafter\ignorespacesandallpars\@gobble}%
    {}%
}
\makeatother
\ifluatex
  \usepackage{selnolig}  % disable illegal ligatures
\fi

\title{R vs Python}
\author{Claude REN}
\date{10/22/2020}

\begin{document}
\maketitle

\hypertarget{introduction}{%
\subsection{Introduction}\label{introduction}}

In this .pdf you will find all the basics of R and its equivalence in
Python. The purpose is to provide a document where you can easily find
useful information when working on R and Python.

\hypertarget{basic-operator}{%
\subsection{Basic operator}\label{basic-operator}}

\begin{column}{0.48\textwidth}

\hypertarget{calculate-with-r}{%
\subsubsection{Calculate with R}\label{calculate-with-r}}

\begin{Shaded}
\begin{Highlighting}[]
\CommentTok{\# No import}
\DecValTok{10}\SpecialCharTok{*}\NormalTok{(}\DecValTok{1}\SpecialCharTok{+}\DecValTok{3}\FloatTok{{-}2.4}\NormalTok{)}
\end{Highlighting}
\end{Shaded}

\begin{verbatim}
[1] 16
\end{verbatim}

\begin{Shaded}
\begin{Highlighting}[]
\DecValTok{10}\SpecialCharTok{\^{}}\DecValTok{2}
\end{Highlighting}
\end{Shaded}

\begin{verbatim}
[1] 100
\end{verbatim}

\begin{Shaded}
\begin{Highlighting}[]
\DecValTok{10}\SpecialCharTok{**}\DecValTok{2}
\end{Highlighting}
\end{Shaded}

\begin{verbatim}
[1] 100
\end{verbatim}

\begin{Shaded}
\begin{Highlighting}[]
\FunctionTok{sqrt}\NormalTok{(}\DecValTok{100}\NormalTok{)}
\end{Highlighting}
\end{Shaded}

\begin{verbatim}
[1] 10
\end{verbatim}

\begin{Shaded}
\begin{Highlighting}[]
\NormalTok{pi}
\end{Highlighting}
\end{Shaded}

\begin{verbatim}
[1] 3.141593
\end{verbatim}

\begin{Shaded}
\begin{Highlighting}[]
\FunctionTok{cos}\NormalTok{(pi)}
\end{Highlighting}
\end{Shaded}

\begin{verbatim}
[1] -1
\end{verbatim}

\begin{Shaded}
\begin{Highlighting}[]
\FunctionTok{exp}\NormalTok{(}\DecValTok{1}\NormalTok{)}
\end{Highlighting}
\end{Shaded}

\begin{verbatim}
[1] 2.718282
\end{verbatim}

\begin{Shaded}
\begin{Highlighting}[]
\FunctionTok{log}\NormalTok{(}\DecValTok{1}\NormalTok{)}
\end{Highlighting}
\end{Shaded}

\begin{verbatim}
[1] 0
\end{verbatim}

\begin{Shaded}
\begin{Highlighting}[]
\FunctionTok{round}\NormalTok{(}\FloatTok{2.5435}\NormalTok{, }\DecValTok{2}\NormalTok{)}
\end{Highlighting}
\end{Shaded}

\begin{verbatim}
[1] 2.54
\end{verbatim}

\begin{Shaded}
\begin{Highlighting}[]
\NormalTok{a }\OtherTok{\textless{}{-}} \DecValTok{100}
\FunctionTok{print}\NormalTok{(a)}
\end{Highlighting}
\end{Shaded}

\begin{verbatim}
[1] 100
\end{verbatim}

\end{column}

\begin{column}{0.04\textwidth}

~

\end{column}

\begin{column}{0.48\textwidth}

\hypertarget{calculate-with-python}{%
\subsubsection{Calculate with Python}\label{calculate-with-python}}

\begin{Shaded}
\begin{Highlighting}[]
\ImportTok{import}\NormalTok{ numpy }\ImportTok{as}\NormalTok{ np}
\DecValTok{10}\OperatorTok{*}\NormalTok{(}\DecValTok{1}\OperatorTok{+}\DecValTok{3}\OperatorTok{{-}}\FloatTok{2.4}\NormalTok{)}
\end{Highlighting}
\end{Shaded}

\begin{verbatim}
16.0
\end{verbatim}

\begin{Shaded}
\begin{Highlighting}[]
\DecValTok{10}\OperatorTok{\^{}}\DecValTok{2}        \CommentTok{\# Not correct !}
\end{Highlighting}
\end{Shaded}

\begin{verbatim}
8
\end{verbatim}

\begin{Shaded}
\begin{Highlighting}[]
\DecValTok{10}\OperatorTok{**}\DecValTok{2}
\end{Highlighting}
\end{Shaded}

\begin{verbatim}
100
\end{verbatim}

\begin{Shaded}
\begin{Highlighting}[]
\NormalTok{np.sqrt(}\DecValTok{100}\NormalTok{)   }
\end{Highlighting}
\end{Shaded}

\begin{verbatim}
10.0
\end{verbatim}

\begin{Shaded}
\begin{Highlighting}[]
\NormalTok{np.pi          }
\end{Highlighting}
\end{Shaded}

\begin{verbatim}
3.141592653589793
\end{verbatim}

\begin{Shaded}
\begin{Highlighting}[]
\NormalTok{np.cos(np.pi)     }
\end{Highlighting}
\end{Shaded}

\begin{verbatim}
-1.0
\end{verbatim}

\begin{Shaded}
\begin{Highlighting}[]
\NormalTok{np.exp(}\DecValTok{1}\NormalTok{)      }
\end{Highlighting}
\end{Shaded}

\begin{verbatim}
2.718281828459045
\end{verbatim}

\begin{Shaded}
\begin{Highlighting}[]
\NormalTok{np.log(}\DecValTok{1}\NormalTok{)      }
\end{Highlighting}
\end{Shaded}

\begin{verbatim}
0.0
\end{verbatim}

\begin{Shaded}
\begin{Highlighting}[]
\BuiltInTok{round}\NormalTok{(}\FloatTok{2.543534}\NormalTok{, }\DecValTok{2}\NormalTok{)}
\end{Highlighting}
\end{Shaded}

\begin{verbatim}
2.54
\end{verbatim}

\begin{Shaded}
\begin{Highlighting}[]
\NormalTok{a }\OperatorTok{=} \DecValTok{100}
\BuiltInTok{print}\NormalTok{(a)}
\end{Highlighting}
\end{Shaded}

\begin{verbatim}
100
\end{verbatim}

\end{column}

\newline

\begin{column}{0.48\textwidth}

\hypertarget{vector-operations-with-r}{%
\subsubsection{Vector operations with
R}\label{vector-operations-with-r}}

\begin{Shaded}
\begin{Highlighting}[]
\NormalTok{v }\OtherTok{\textless{}{-}} \FunctionTok{c}\NormalTok{(}\DecValTok{10}\NormalTok{, }\DecValTok{20}\NormalTok{, }\DecValTok{30}\NormalTok{)}
\NormalTok{v}
\end{Highlighting}
\end{Shaded}

\begin{verbatim}
[1] 10 20 30
\end{verbatim}

\begin{Shaded}
\begin{Highlighting}[]
\FunctionTok{length}\NormalTok{(v)}
\end{Highlighting}
\end{Shaded}

\begin{verbatim}
[1] 3
\end{verbatim}

\begin{Shaded}
\begin{Highlighting}[]
\DecValTok{2}\SpecialCharTok{*}\NormalTok{v}\SpecialCharTok{+}\DecValTok{1}
\end{Highlighting}
\end{Shaded}

\begin{verbatim}
[1] 21 41 61
\end{verbatim}

\begin{Shaded}
\begin{Highlighting}[]
\NormalTok{v}\SpecialCharTok{**}\DecValTok{2}
\end{Highlighting}
\end{Shaded}

\begin{verbatim}
[1] 100 400 900
\end{verbatim}

\begin{Shaded}
\begin{Highlighting}[]
\FunctionTok{log}\NormalTok{(v)}
\end{Highlighting}
\end{Shaded}

\begin{verbatim}
[1] 2.302585 2.995732 3.401197
\end{verbatim}

\begin{Shaded}
\begin{Highlighting}[]
\NormalTok{w }\OtherTok{\textless{}{-}} \FunctionTok{c}\NormalTok{(}\DecValTok{1}\NormalTok{, }\DecValTok{2}\NormalTok{, }\DecValTok{3}\NormalTok{)}
\NormalTok{v}\SpecialCharTok{{-}}\NormalTok{w}
\end{Highlighting}
\end{Shaded}

\begin{verbatim}
[1]  9 18 27
\end{verbatim}

\begin{Shaded}
\begin{Highlighting}[]
\NormalTok{v}\SpecialCharTok{*}\NormalTok{w}
\end{Highlighting}
\end{Shaded}

\begin{verbatim}
[1] 10 40 90
\end{verbatim}

\begin{Shaded}
\begin{Highlighting}[]
\NormalTok{v}\SpecialCharTok{/}\NormalTok{w}
\end{Highlighting}
\end{Shaded}

\begin{verbatim}
[1] 10 10 10
\end{verbatim}

\begin{Shaded}
\begin{Highlighting}[]
\NormalTok{v}\SpecialCharTok{\%*\%}\NormalTok{w}
\end{Highlighting}
\end{Shaded}

\begin{verbatim}
     [,1]
[1,]  140
\end{verbatim}

\begin{Shaded}
\begin{Highlighting}[]
\FunctionTok{sum}\NormalTok{(v)}
\end{Highlighting}
\end{Shaded}

\begin{verbatim}
[1] 60
\end{verbatim}

\begin{Shaded}
\begin{Highlighting}[]
\FunctionTok{mean}\NormalTok{(v)}
\end{Highlighting}
\end{Shaded}

\begin{verbatim}
[1] 20
\end{verbatim}

\begin{Shaded}
\begin{Highlighting}[]
\FunctionTok{min}\NormalTok{(v)}
\end{Highlighting}
\end{Shaded}

\begin{verbatim}
[1] 10
\end{verbatim}

\begin{Shaded}
\begin{Highlighting}[]
\FunctionTok{max}\NormalTok{(v)}
\end{Highlighting}
\end{Shaded}

\begin{verbatim}
[1] 30
\end{verbatim}

\begin{Shaded}
\begin{Highlighting}[]
\FunctionTok{sd}\NormalTok{(v)}
\end{Highlighting}
\end{Shaded}

\begin{verbatim}
[1] 10
\end{verbatim}

\begin{Shaded}
\begin{Highlighting}[]
\FunctionTok{median}\NormalTok{(v)}
\end{Highlighting}
\end{Shaded}

\begin{verbatim}
[1] 20
\end{verbatim}

\end{column}

\begin{column}{0.04\textwidth}

~

\end{column}

\begin{column}{0.48\textwidth}

\hypertarget{vector-operations-with-python}{%
\subsubsection{Vector operations with
Python}\label{vector-operations-with-python}}

\begin{Shaded}
\begin{Highlighting}[]
\NormalTok{v }\OperatorTok{=}\NormalTok{ np.array([}\DecValTok{10}\NormalTok{, }\DecValTok{20}\NormalTok{, }\DecValTok{30}\NormalTok{])}
\NormalTok{v}
\end{Highlighting}
\end{Shaded}

\begin{verbatim}
array([10, 20, 30])
\end{verbatim}

\begin{Shaded}
\begin{Highlighting}[]
\BuiltInTok{len}\NormalTok{(v)}
\end{Highlighting}
\end{Shaded}

\begin{verbatim}
3
\end{verbatim}

\begin{Shaded}
\begin{Highlighting}[]
\DecValTok{2}\OperatorTok{*}\NormalTok{v}\OperatorTok{+}\DecValTok{1}
\end{Highlighting}
\end{Shaded}

\begin{verbatim}
array([21, 41, 61])
\end{verbatim}

\begin{Shaded}
\begin{Highlighting}[]
\NormalTok{v}\OperatorTok{**}\DecValTok{2}
\end{Highlighting}
\end{Shaded}

\begin{verbatim}
array([100, 400, 900])
\end{verbatim}

\begin{Shaded}
\begin{Highlighting}[]
\NormalTok{np.log(v).}\BuiltInTok{round}\NormalTok{(}\DecValTok{4}\NormalTok{)}
\end{Highlighting}
\end{Shaded}

\begin{verbatim}
array([2.3026, 2.9957, 3.4012])
\end{verbatim}

\begin{Shaded}
\begin{Highlighting}[]
\NormalTok{w }\OperatorTok{=}\NormalTok{ np.array([}\DecValTok{1}\NormalTok{, }\DecValTok{2}\NormalTok{, }\DecValTok{3}\NormalTok{])}
\NormalTok{v}\OperatorTok{{-}}\NormalTok{w}
\end{Highlighting}
\end{Shaded}

\begin{verbatim}
array([ 9, 18, 27])
\end{verbatim}

\begin{Shaded}
\begin{Highlighting}[]
\NormalTok{v}\OperatorTok{*}\NormalTok{w}
\end{Highlighting}
\end{Shaded}

\begin{verbatim}
array([10, 40, 90])
\end{verbatim}

\begin{Shaded}
\begin{Highlighting}[]
\NormalTok{v}\OperatorTok{/}\NormalTok{w}
\end{Highlighting}
\end{Shaded}

\begin{verbatim}
array([10., 10., 10.])
\end{verbatim}

\begin{Shaded}
\begin{Highlighting}[]
\NormalTok{np.dot(v,w)}
\end{Highlighting}
\end{Shaded}

\begin{verbatim}
140
\end{verbatim}

\begin{Shaded}
\begin{Highlighting}[]
\BuiltInTok{sum}\NormalTok{(v)}
\end{Highlighting}
\end{Shaded}

\begin{verbatim}
60
\end{verbatim}

\begin{Shaded}
\begin{Highlighting}[]
\NormalTok{np.average(v)}
\end{Highlighting}
\end{Shaded}

\begin{verbatim}
20.0
\end{verbatim}

\begin{Shaded}
\begin{Highlighting}[]
\BuiltInTok{min}\NormalTok{(v)}
\end{Highlighting}
\end{Shaded}

\begin{verbatim}
10
\end{verbatim}

\begin{Shaded}
\begin{Highlighting}[]
\BuiltInTok{max}\NormalTok{(v)}
\end{Highlighting}
\end{Shaded}

\begin{verbatim}
30
\end{verbatim}

\begin{Shaded}
\begin{Highlighting}[]
\NormalTok{np.std(v, ddof }\OperatorTok{=} \DecValTok{1}\NormalTok{)}
\end{Highlighting}
\end{Shaded}

\begin{verbatim}
10.0
\end{verbatim}

\begin{Shaded}
\begin{Highlighting}[]
\NormalTok{np.median(v)}
\end{Highlighting}
\end{Shaded}

\begin{verbatim}
20.0
\end{verbatim}

\end{column}

\newline

For the standard deviation, the formula used in numpy with std() is
different from the one in R with sd().

\begin{center}

$\sigma_p = \sqrt{\frac{1}{n}\sum_{i=1}^n (x_i-\overline{x})^2} = \sqrt{\frac{1}{n}\sum_{i=1}^n x_i^2 - \overline{x}^2}$

$\sigma_R = \sqrt{\frac{1}{n-1}\sum_{i=1}^n (x_i-\overline{x})^2} = \sqrt{\frac{1}{n-1}\sum_{i=1}^n x_i^2 - \overline{x}^2}$

\end{center}

In order to find the same result in Python as in R, you have to precise
ddof = 1 in Python.

\begin{column}{0.48\textwidth}

\hypertarget{vector-manipulation-with-r}{%
\subsubsection{Vector manipulation with
R}\label{vector-manipulation-with-r}}

\begin{Shaded}
\begin{Highlighting}[]
\NormalTok{u }\OtherTok{\textless{}{-}} \FunctionTok{c}\NormalTok{(}\DecValTok{1}\NormalTok{, }\DecValTok{2}\NormalTok{, }\DecValTok{3}\NormalTok{, }\DecValTok{4}\NormalTok{, }\DecValTok{5}\NormalTok{)}
\NormalTok{u[}\DecValTok{2}\NormalTok{]}
\end{Highlighting}
\end{Shaded}

\begin{verbatim}
[1] 2
\end{verbatim}

\begin{Shaded}
\begin{Highlighting}[]
\NormalTok{u[}\DecValTok{3}\SpecialCharTok{:}\DecValTok{5}\NormalTok{]}
\end{Highlighting}
\end{Shaded}

\begin{verbatim}
[1] 3 4 5
\end{verbatim}

\begin{Shaded}
\begin{Highlighting}[]
\NormalTok{u[}\DecValTok{5}\NormalTok{] }\OtherTok{\textless{}{-}} \DecValTok{50}
\NormalTok{u[}\DecValTok{1}\SpecialCharTok{:}\DecValTok{3}\NormalTok{] }\OtherTok{\textless{}{-}} \DecValTok{1}
\NormalTok{u}
\end{Highlighting}
\end{Shaded}

\begin{verbatim}
[1]  1  1  1  4 50
\end{verbatim}

\begin{Shaded}
\begin{Highlighting}[]
\NormalTok{v }\OtherTok{\textless{}{-}} \FunctionTok{c}\NormalTok{(}\DecValTok{10}\NormalTok{,}\DecValTok{20}\NormalTok{,}\DecValTok{30}\NormalTok{,}\DecValTok{30}\NormalTok{,}\DecValTok{60}\NormalTok{,}\DecValTok{50}\NormalTok{)}
\NormalTok{w }\OtherTok{\textless{}{-}} \FunctionTok{c}\NormalTok{(}\DecValTok{20}\NormalTok{,}\DecValTok{10}\NormalTok{,}\DecValTok{31}\NormalTok{,}\DecValTok{31}\NormalTok{,}\DecValTok{61}\NormalTok{,}\DecValTok{51}\NormalTok{)}
\NormalTok{u }\OtherTok{\textless{}{-}} \FunctionTok{c}\NormalTok{(}\DecValTok{5}\NormalTok{ ,}\DecValTok{5}\NormalTok{ ,}\DecValTok{5}\NormalTok{ ,}\DecValTok{32}\NormalTok{,}\DecValTok{62}\NormalTok{,}\DecValTok{49}\NormalTok{)}
\end{Highlighting}
\end{Shaded}

\end{column}

\begin{column}{0.04\textwidth}

~

\end{column}

\begin{column}{0.48\textwidth}

\hypertarget{vector-manipulation-with-python}{%
\subsubsection{Vector manipulation with
Python}\label{vector-manipulation-with-python}}

\begin{Shaded}
\begin{Highlighting}[]
\NormalTok{u }\OperatorTok{=}\NormalTok{ np.array([}\DecValTok{1}\NormalTok{, }\DecValTok{2}\NormalTok{, }\DecValTok{3}\NormalTok{, }\DecValTok{4}\NormalTok{, }\DecValTok{5}\NormalTok{])}
\NormalTok{u[}\DecValTok{1}\NormalTok{]}
\end{Highlighting}
\end{Shaded}

\begin{verbatim}
2
\end{verbatim}

\begin{Shaded}
\begin{Highlighting}[]
\NormalTok{u[}\DecValTok{2}\NormalTok{:}\DecValTok{5}\NormalTok{]}
\end{Highlighting}
\end{Shaded}

\begin{verbatim}
array([3, 4, 5])
\end{verbatim}

\begin{Shaded}
\begin{Highlighting}[]
\NormalTok{u[}\DecValTok{4}\NormalTok{] }\OperatorTok{=} \DecValTok{50}
\NormalTok{u[}\DecValTok{0}\NormalTok{:}\DecValTok{3}\NormalTok{] }\OperatorTok{=} \DecValTok{1}
\NormalTok{u}
\end{Highlighting}
\end{Shaded}

\begin{verbatim}
array([ 1,  1,  1,  4, 50])
\end{verbatim}

\begin{Shaded}
\begin{Highlighting}[]
\NormalTok{v }\OperatorTok{=}\NormalTok{ np.array([}\DecValTok{10}\NormalTok{,}\DecValTok{20}\NormalTok{,}\DecValTok{30}\NormalTok{,}\DecValTok{30}\NormalTok{,}\DecValTok{60}\NormalTok{,}\DecValTok{50}\NormalTok{])}
\NormalTok{w }\OperatorTok{=}\NormalTok{ np.array([}\DecValTok{20}\NormalTok{,}\DecValTok{10}\NormalTok{,}\DecValTok{31}\NormalTok{,}\DecValTok{31}\NormalTok{,}\DecValTok{61}\NormalTok{,}\DecValTok{51}\NormalTok{])}
\NormalTok{u }\OperatorTok{=}\NormalTok{ np.array([}\DecValTok{5}\NormalTok{ ,}\DecValTok{5}\NormalTok{ ,}\DecValTok{5}\NormalTok{ ,}\DecValTok{32}\NormalTok{,}\DecValTok{62}\NormalTok{,}\DecValTok{49}\NormalTok{])}
\end{Highlighting}
\end{Shaded}

\end{column}

\newline

\hypertarget{equivalence-str-r-in-python}{%
\paragraph{Equivalence str() R in Python
:}\label{equivalence-str-r-in-python}}

The function str() in R show you the structure of your variable, it can
be very useful and you'll will find no equivalent function in Python.
The only way to get the same information is to create a function :

\begin{Shaded}
\begin{Highlighting}[]
\FunctionTok{str}\NormalTok{(u)}
\end{Highlighting}
\end{Shaded}

\begin{verbatim}
 num [1:6] 5 5 5 32 62 49
\end{verbatim}

\begin{Shaded}
\begin{Highlighting}[]
\ImportTok{import}\NormalTok{ pandas }\ImportTok{as}\NormalTok{ pd}
\NormalTok{du }\OperatorTok{=}\NormalTok{ pd.DataFrame(u)}
\BuiltInTok{print}\NormalTok{(}\BuiltInTok{str}\NormalTok{(du.info()) }\OperatorTok{+} \StringTok{"}\CharTok{\textbackslash{}n}\StringTok{"}\OperatorTok{+} \BuiltInTok{str}\NormalTok{(u))}
\end{Highlighting}
\end{Shaded}

\begin{verbatim}
<class 'pandas.core.frame.DataFrame'>
RangeIndex: 6 entries, 0 to 5
Data columns (total 1 columns):
 #   Column  Non-Null Count  Dtype
---  ------  --------------  -----
 0   0       6 non-null      int64
dtypes: int64(1)
memory usage: 176.0 bytes
None
[ 5  5  5 32 62 49]
\end{verbatim}

You can see above that the str() function used in the python chunks is
not at all the same function as in R. In Python it is used to transform
non string type to string character. \newline \newline The one below is
for data.frame, keep it for future usage :

\begin{Shaded}
\begin{Highlighting}[]
\KeywordTok{def}\NormalTok{ rstr(df): }
\NormalTok{    structural\_info }\OperatorTok{=}\NormalTok{ pd.DataFrame(index}\OperatorTok{=}\NormalTok{df.columns)}
\NormalTok{    structural\_info[}\StringTok{\textquotesingle{}unique\_len\textquotesingle{}}\NormalTok{] }\OperatorTok{=}\NormalTok{ df.}\BuiltInTok{apply}\NormalTok{(}\KeywordTok{lambda}\NormalTok{ x: }\BuiltInTok{len}\NormalTok{(x.unique())).values}
\NormalTok{    structural\_info[}\StringTok{\textquotesingle{}unique\_val\textquotesingle{}}\NormalTok{] }\OperatorTok{=}\NormalTok{ df.}\BuiltInTok{apply}\NormalTok{(}\KeywordTok{lambda}\NormalTok{ x: [x.unique()]).values}
    \BuiltInTok{print}\NormalTok{(df.shape)}
    \ControlFlowTok{return}\NormalTok{ structural\_info  }
\NormalTok{rstr(df)}
\end{Highlighting}
\end{Shaded}

\pagebreak

\begin{column}{0.48\textwidth}

\hypertarget{vector-manipulation-with-r-1}{%
\subsubsection{Vector manipulation with
R}\label{vector-manipulation-with-r-1}}

\begin{Shaded}
\begin{Highlighting}[]
\FunctionTok{options}\NormalTok{(}\AttributeTok{width =} \DecValTok{30}\NormalTok{)}
\FunctionTok{sum}\NormalTok{(}\FunctionTok{is.na}\NormalTok{(u))}
\end{Highlighting}
\end{Shaded}

\begin{verbatim}
[1] 0
\end{verbatim}

\begin{Shaded}
\begin{Highlighting}[]
\NormalTok{u\_ }\OtherTok{\textless{}{-}} \FunctionTok{c}\NormalTok{(}\ConstantTok{NA}\NormalTok{,u,}\ConstantTok{NA}\NormalTok{,}\ConstantTok{NA}\NormalTok{)}
\NormalTok{u\_}
\end{Highlighting}
\end{Shaded}

\begin{verbatim}
[1] NA  5  5  5 32 62 49 NA NA
\end{verbatim}

\begin{Shaded}
\begin{Highlighting}[]
\FunctionTok{sum}\NormalTok{(}\FunctionTok{is.na}\NormalTok{(u\_))}
\end{Highlighting}
\end{Shaded}

\begin{verbatim}
[1] 3
\end{verbatim}

\begin{Shaded}
\begin{Highlighting}[]
\FunctionTok{range}\NormalTok{(u)}
\end{Highlighting}
\end{Shaded}

\begin{verbatim}
[1]  5 62
\end{verbatim}

\begin{Shaded}
\begin{Highlighting}[]
\FunctionTok{range}\NormalTok{(u\_ , }\AttributeTok{na.rm =} \ConstantTok{TRUE}\NormalTok{)}
\end{Highlighting}
\end{Shaded}

\begin{verbatim}
[1]  5 62
\end{verbatim}

\begin{Shaded}
\begin{Highlighting}[]
\FunctionTok{quantile}\NormalTok{(u)}
\end{Highlighting}
\end{Shaded}

\begin{verbatim}
   0%   25%   50%   75%  100% 
 5.00  5.00 18.50 44.75 62.00 
\end{verbatim}

\begin{Shaded}
\begin{Highlighting}[]
\FunctionTok{summary}\NormalTok{(u)}
\end{Highlighting}
\end{Shaded}

\begin{verbatim}
   Min. 1st Qu.  Median 
   5.00    5.00   18.50 
   Mean 3rd Qu.    Max. 
  26.33   44.75   62.00 
\end{verbatim}

\begin{Shaded}
\begin{Highlighting}[]
\FunctionTok{sd}\NormalTok{(u\_, }\AttributeTok{na.rm =} \ConstantTok{TRUE}\NormalTok{)}
\end{Highlighting}
\end{Shaded}

\begin{verbatim}
[1] 25.23225
\end{verbatim}

\begin{Shaded}
\begin{Highlighting}[]
\FunctionTok{cor}\NormalTok{(v,w)}
\end{Highlighting}
\end{Shaded}

\begin{verbatim}
[1] 0.9433573
\end{verbatim}

\begin{Shaded}
\begin{Highlighting}[]
\FunctionTok{sort}\NormalTok{(v)}
\end{Highlighting}
\end{Shaded}

\begin{verbatim}
[1] 10 20 30 30 50 60
\end{verbatim}

\begin{Shaded}
\begin{Highlighting}[]
\FunctionTok{sort}\NormalTok{(v, }\AttributeTok{decreasing =} \ConstantTok{TRUE}\NormalTok{)}
\end{Highlighting}
\end{Shaded}

\begin{verbatim}
[1] 60 50 30 30 20 10
\end{verbatim}

\begin{Shaded}
\begin{Highlighting}[]
\FunctionTok{order}\NormalTok{(w)}
\end{Highlighting}
\end{Shaded}

\begin{verbatim}
[1] 2 1 3 4 6 5
\end{verbatim}

\begin{Shaded}
\begin{Highlighting}[]
\FunctionTok{rank}\NormalTok{(w, }\AttributeTok{ties.method=}\StringTok{"min"}\NormalTok{)}
\end{Highlighting}
\end{Shaded}

\begin{verbatim}
[1] 2 1 3 3 6 5
\end{verbatim}

\begin{Shaded}
\begin{Highlighting}[]
\FunctionTok{rank}\NormalTok{(w, }\AttributeTok{ties.method=}\StringTok{"max"}\NormalTok{)}
\end{Highlighting}
\end{Shaded}

\begin{verbatim}
[1] 2 1 4 4 6 5
\end{verbatim}

\begin{Shaded}
\begin{Highlighting}[]
\FunctionTok{pmax}\NormalTok{(v,w,u)}
\end{Highlighting}
\end{Shaded}

\begin{verbatim}
[1] 20 20 31 32 62 51
\end{verbatim}

\begin{Shaded}
\begin{Highlighting}[]
\FunctionTok{pmin}\NormalTok{(v,w,u)}
\end{Highlighting}
\end{Shaded}

\begin{verbatim}
[1]  5  5  5 30 60 49
\end{verbatim}

\end{column}

\begin{column}{0.04\textwidth}

~

\end{column}

\begin{column}{0.48\textwidth}

\hypertarget{vector-manipulation-with-python-1}{%
\subsubsection{Vector manipulation with
Python}\label{vector-manipulation-with-python-1}}

\begin{Shaded}
\begin{Highlighting}[]
\NormalTok{np.set\_printoptions(}
\NormalTok{    suppress}\OperatorTok{=}\VariableTok{True}\NormalTok{,linewidth}\OperatorTok{=}\DecValTok{40}\NormalTok{)}
\NormalTok{du.isna().}\BuiltInTok{sum}\NormalTok{()}
\end{Highlighting}
\end{Shaded}

\begin{verbatim}
0    0
dtype: int64
\end{verbatim}

\begin{Shaded}
\begin{Highlighting}[]
\NormalTok{u\_ }\OperatorTok{=}\NormalTok{ np.append(u,np.nan)}
\NormalTok{u\_ }\OperatorTok{=}\NormalTok{ np.append(np.nan, u\_)}
\NormalTok{u\_ }\OperatorTok{=}\NormalTok{ np.append(u\_, np.nan)}
\NormalTok{u\_}
\end{Highlighting}
\end{Shaded}

\begin{verbatim}
array([nan,  5.,  5.,  5., 32., 62.,
       49., nan, nan])
\end{verbatim}

\begin{Shaded}
\begin{Highlighting}[]
\NormalTok{du\_ }\OperatorTok{=}\NormalTok{ pd.DataFrame(u\_)}
\NormalTok{du\_.isna().}\BuiltInTok{sum}\NormalTok{()}
\end{Highlighting}
\end{Shaded}

\begin{verbatim}
0    3
dtype: int64
\end{verbatim}

\begin{Shaded}
\begin{Highlighting}[]
\BuiltInTok{print}\NormalTok{(}\BuiltInTok{str}\NormalTok{(}\BuiltInTok{min}\NormalTok{(u)) }\OperatorTok{+} \StringTok{" "} \OperatorTok{+} \BuiltInTok{str}\NormalTok{(}\BuiltInTok{max}\NormalTok{(u)))}
\end{Highlighting}
\end{Shaded}

\begin{verbatim}
5 62
\end{verbatim}

\begin{Shaded}
\begin{Highlighting}[]
\BuiltInTok{print}\NormalTok{(}\BuiltInTok{str}\NormalTok{(np.nanmin(u\_)) }\OperatorTok{+} 
  \StringTok{" "} \OperatorTok{+} \BuiltInTok{str}\NormalTok{(np.nanmax(u\_)))}
\end{Highlighting}
\end{Shaded}

\begin{verbatim}
5.0 62.0
\end{verbatim}

\begin{Shaded}
\begin{Highlighting}[]
\NormalTok{np.quantile(u, q }\OperatorTok{=} \FloatTok{0.5}\NormalTok{) }\CommentTok{\# q represent the \%}
\end{Highlighting}
\end{Shaded}

\begin{verbatim}
18.5
\end{verbatim}

\begin{Shaded}
\begin{Highlighting}[]
\NormalTok{du.describe()   }\CommentTok{\# It works like summary()}
\end{Highlighting}
\end{Shaded}

\begin{Shaded}
\begin{Highlighting}[]
\NormalTok{np.nanstd(u\_, ddof }\OperatorTok{=} \DecValTok{1}\NormalTok{)}
\end{Highlighting}
\end{Shaded}

\begin{verbatim}
25.232254490367417
\end{verbatim}

\begin{Shaded}
\begin{Highlighting}[]
\NormalTok{np.corrcoef(v,w)}
\end{Highlighting}
\end{Shaded}

\begin{verbatim}
array([[1.       , 0.9433573],
       [0.9433573, 1.       ]])
\end{verbatim}

\begin{Shaded}
\begin{Highlighting}[]
\NormalTok{np.sort(v)}
\end{Highlighting}
\end{Shaded}

\begin{verbatim}
array([10, 20, 30, 30, 50, 60])
\end{verbatim}

\begin{Shaded}
\begin{Highlighting}[]
\OperatorTok{{-}}\NormalTok{np.sort(}\OperatorTok{{-}}\NormalTok{v)}
\end{Highlighting}
\end{Shaded}

\begin{verbatim}
array([60, 50, 30, 30, 20, 10])
\end{verbatim}

\begin{Shaded}
\begin{Highlighting}[]
\NormalTok{np.argsort(w)}
\end{Highlighting}
\end{Shaded}

\begin{verbatim}
array([1, 0, 2, 3, 5, 4])
\end{verbatim}

\begin{Shaded}
\begin{Highlighting}[]
\NormalTok{[}\BuiltInTok{sorted}\NormalTok{(w).index(x) }\ControlFlowTok{for}\NormalTok{ x }\KeywordTok{in}\NormalTok{ w]}
\CommentTok{\# No easy equivalence for the "max"}
\end{Highlighting}
\end{Shaded}

\begin{verbatim}
[1, 0, 2, 2, 5, 4]
\end{verbatim}

\begin{Shaded}
\begin{Highlighting}[]
\NormalTok{z }\OperatorTok{=}\NormalTok{ np.maximum(u,v)}
\NormalTok{np.maximum(z,w) }\CommentTok{\# Same with minimum()}
\end{Highlighting}
\end{Shaded}

\begin{verbatim}
array([20, 20, 31, 32, 62, 51])
\end{verbatim}

\end{column}

\newline

\begin{column}{0.48\textwidth}

\hypertarget{vector-manipulation-with-r-2}{%
\subsubsection{Vector manipulation with
R}\label{vector-manipulation-with-r-2}}

\begin{Shaded}
\begin{Highlighting}[]
\FunctionTok{options}\NormalTok{(}\AttributeTok{width =} \DecValTok{30}\NormalTok{)}
\FunctionTok{cumsum}\NormalTok{(v)}
\end{Highlighting}
\end{Shaded}

\begin{verbatim}
[1]  10  30  60  90 150 200
\end{verbatim}

\begin{Shaded}
\begin{Highlighting}[]
\FunctionTok{cumprod}\NormalTok{(v)}
\end{Highlighting}
\end{Shaded}

\begin{verbatim}
[1] 1.00e+01 2.00e+02 6.00e+03
[4] 1.80e+05 1.08e+07 5.40e+08
\end{verbatim}

\begin{Shaded}
\begin{Highlighting}[]
\FunctionTok{cummax}\NormalTok{(w)}
\end{Highlighting}
\end{Shaded}

\begin{verbatim}
[1] 20 20 31 31 61 61
\end{verbatim}

\begin{Shaded}
\begin{Highlighting}[]
\FunctionTok{cummin}\NormalTok{(w)}
\end{Highlighting}
\end{Shaded}

\begin{verbatim}
[1] 20 10 10 10 10 10
\end{verbatim}

\hypertarget{boolean-operation-with-r}{%
\subsubsection{Boolean operation with
R}\label{boolean-operation-with-r}}

\begin{Shaded}
\begin{Highlighting}[]
\NormalTok{a }\OtherTok{\textless{}{-}} \DecValTok{1} 
\NormalTok{b }\OtherTok{\textless{}{-}} \DecValTok{2}
\NormalTok{(a }\SpecialCharTok{==} \DecValTok{1}\NormalTok{)}
\end{Highlighting}
\end{Shaded}

\begin{verbatim}
[1] TRUE
\end{verbatim}

\begin{Shaded}
\begin{Highlighting}[]
\NormalTok{(a }\SpecialCharTok{==}\NormalTok{ b)}
\end{Highlighting}
\end{Shaded}

\begin{verbatim}
[1] FALSE
\end{verbatim}

\begin{Shaded}
\begin{Highlighting}[]
\NormalTok{(a }\SpecialCharTok{\textless{}=}\NormalTok{ b)}
\end{Highlighting}
\end{Shaded}

\begin{verbatim}
[1] TRUE
\end{verbatim}

\begin{Shaded}
\begin{Highlighting}[]
\NormalTok{A }\OtherTok{\textless{}{-}} \FunctionTok{c}\NormalTok{(}\ConstantTok{TRUE}\NormalTok{,}\ConstantTok{TRUE}\NormalTok{,}\ConstantTok{FALSE}\NormalTok{,}\ConstantTok{FALSE}\NormalTok{)}
\NormalTok{B }\OtherTok{\textless{}{-}} \FunctionTok{c}\NormalTok{(}\ConstantTok{TRUE}\NormalTok{,}\ConstantTok{FALSE}\NormalTok{,}\ConstantTok{TRUE}\NormalTok{,}\ConstantTok{FALSE}\NormalTok{)}
\NormalTok{A }\SpecialCharTok{\&}\NormalTok{ B}
\end{Highlighting}
\end{Shaded}

\begin{verbatim}
[1]  TRUE FALSE FALSE FALSE
\end{verbatim}

\begin{Shaded}
\begin{Highlighting}[]
\NormalTok{A }\SpecialCharTok{|}\NormalTok{ B}
\end{Highlighting}
\end{Shaded}

\begin{verbatim}
[1]  TRUE  TRUE  TRUE FALSE
\end{verbatim}

\begin{Shaded}
\begin{Highlighting}[]
\SpecialCharTok{!}\NormalTok{A}
\end{Highlighting}
\end{Shaded}

\begin{verbatim}
[1] FALSE FALSE  TRUE  TRUE
\end{verbatim}

\begin{Shaded}
\begin{Highlighting}[]
\NormalTok{c }\OtherTok{\textless{}{-}}\NormalTok{ (a }\SpecialCharTok{\textgreater{}}\NormalTok{ b)}
\NormalTok{c}
\end{Highlighting}
\end{Shaded}

\begin{verbatim}
[1] FALSE
\end{verbatim}

\begin{Shaded}
\begin{Highlighting}[]
\NormalTok{v }\OtherTok{\textless{}{-}} \FunctionTok{c}\NormalTok{(}\DecValTok{10}\NormalTok{,}\DecValTok{20}\NormalTok{,}\DecValTok{30}\NormalTok{,}\DecValTok{30}\NormalTok{,}\DecValTok{60}\NormalTok{,}\DecValTok{50}\NormalTok{)}
\NormalTok{t }\OtherTok{\textless{}{-}}\NormalTok{ (v }\SpecialCharTok{\textgreater{}} \DecValTok{30}\NormalTok{)}
\NormalTok{t}
\end{Highlighting}
\end{Shaded}

\begin{verbatim}
[1] FALSE FALSE FALSE FALSE
[5]  TRUE  TRUE
\end{verbatim}

\begin{Shaded}
\begin{Highlighting}[]
\NormalTok{w }\OtherTok{\textless{}{-}}\NormalTok{ v[(v}\SpecialCharTok{\textgreater{}}\DecValTok{30}\NormalTok{)]}
\NormalTok{w}
\end{Highlighting}
\end{Shaded}

\begin{verbatim}
[1] 60 50
\end{verbatim}

\begin{Shaded}
\begin{Highlighting}[]
\FunctionTok{which}\NormalTok{(v }\SpecialCharTok{==} \DecValTok{30}\NormalTok{)}
\end{Highlighting}
\end{Shaded}

\begin{verbatim}
[1] 3 4
\end{verbatim}

\end{column}

\begin{column}{0.04\textwidth}

~

\end{column}

\begin{column}{0.48\textwidth}

\hypertarget{vector-manipulation-with-python-2}{%
\subsubsection{Vector manipulation with
Python}\label{vector-manipulation-with-python-2}}

\begin{Shaded}
\begin{Highlighting}[]
\NormalTok{np.cumsum(v)}
\end{Highlighting}
\end{Shaded}

\begin{verbatim}
array([ 10,  30,  60,  90, 150, 200])
\end{verbatim}

\begin{Shaded}
\begin{Highlighting}[]
\NormalTok{np.cumprod(v)}
\end{Highlighting}
\end{Shaded}

\begin{verbatim}
array([       10,       200,      6000,
          180000,  10800000, 540000000])
\end{verbatim}

\begin{Shaded}
\begin{Highlighting}[]
\NormalTok{dw }\OperatorTok{=}\NormalTok{ pd.DataFrame(w)}
\NormalTok{dw.cummax().transpose()}
\end{Highlighting}
\end{Shaded}

\begin{verbatim}
    0   1   2   3   4   5
0  20  20  31  31  61  61
\end{verbatim}

\begin{Shaded}
\begin{Highlighting}[]
\NormalTok{dw.cummin().transpose()}
\end{Highlighting}
\end{Shaded}

\begin{verbatim}
    0   1   2   3   4   5
0  20  10  10  10  10  10
\end{verbatim}

\hypertarget{boolean-operations-with-python}{%
\subsubsection{Boolean operations with
Python}\label{boolean-operations-with-python}}

\begin{Shaded}
\begin{Highlighting}[]
\BuiltInTok{bool}\NormalTok{(a }\OperatorTok{==} \DecValTok{1}\NormalTok{)}
\end{Highlighting}
\end{Shaded}

\begin{verbatim}
True
\end{verbatim}

\begin{Shaded}
\begin{Highlighting}[]
\BuiltInTok{bool}\NormalTok{(a }\OperatorTok{==}\NormalTok{ b)}
\end{Highlighting}
\end{Shaded}

\begin{verbatim}
False
\end{verbatim}

\begin{Shaded}
\begin{Highlighting}[]
\BuiltInTok{bool}\NormalTok{(a }\OperatorTok{\textless{}=}\NormalTok{ b)}
\end{Highlighting}
\end{Shaded}

\begin{verbatim}
True
\end{verbatim}

\begin{Shaded}
\begin{Highlighting}[]
\NormalTok{A }\OperatorTok{=}\NormalTok{ np.array([}\VariableTok{True}\NormalTok{,}\VariableTok{True}\NormalTok{,}\VariableTok{False}\NormalTok{,}\VariableTok{False}\NormalTok{])}
\NormalTok{B }\OperatorTok{=}\NormalTok{ np.array([}\VariableTok{True}\NormalTok{,}\VariableTok{False}\NormalTok{,}\VariableTok{True}\NormalTok{,}\VariableTok{False}\NormalTok{])}
\NormalTok{np.logical\_and(A, B)}
\end{Highlighting}
\end{Shaded}

\begin{verbatim}
array([ True, False, False, False])
\end{verbatim}

\begin{Shaded}
\begin{Highlighting}[]
\NormalTok{np.logical\_or(A, B)}
\end{Highlighting}
\end{Shaded}

\begin{verbatim}
array([ True,  True,  True, False])
\end{verbatim}

\begin{Shaded}
\begin{Highlighting}[]
\NormalTok{np.logical\_not(A)}
\end{Highlighting}
\end{Shaded}

\begin{verbatim}
array([False, False,  True,  True])
\end{verbatim}

\begin{Shaded}
\begin{Highlighting}[]
\NormalTok{c }\OperatorTok{=}\NormalTok{ np.array(a }\OperatorTok{\textgreater{}}\NormalTok{ b)}
\NormalTok{c}
\end{Highlighting}
\end{Shaded}

\begin{verbatim}
array(False)
\end{verbatim}

\begin{Shaded}
\begin{Highlighting}[]
\NormalTok{v }\OperatorTok{=}\NormalTok{ np.array([}\DecValTok{10}\NormalTok{,}\DecValTok{20}\NormalTok{,}\DecValTok{30}\NormalTok{,}\DecValTok{30}\NormalTok{,}\DecValTok{60}\NormalTok{,}\DecValTok{50}\NormalTok{])}
\NormalTok{t }\OperatorTok{=}\NormalTok{ np.array([v }\OperatorTok{\textgreater{}} \DecValTok{30}\NormalTok{])}
\NormalTok{t}
\end{Highlighting}
\end{Shaded}

\begin{verbatim}
array([[False, False, False, False,
         True,  True]])
\end{verbatim}

\begin{Shaded}
\begin{Highlighting}[]
\NormalTok{w }\OperatorTok{=}\NormalTok{ v[(v}\OperatorTok{\textgreater{}}\DecValTok{30}\NormalTok{)]}
\NormalTok{w}
\end{Highlighting}
\end{Shaded}

\begin{verbatim}
array([60, 50])
\end{verbatim}

\begin{Shaded}
\begin{Highlighting}[]
\NormalTok{np.where(v }\OperatorTok{==} \DecValTok{30}\NormalTok{)}
\end{Highlighting}
\end{Shaded}

\begin{verbatim}
(array([2, 3]),)
\end{verbatim}

\end{column}

\newline

\begin{column}{0.48\textwidth}

\hypertarget{boolean-operations-with-r}{%
\subsubsection{Boolean operations with
R}\label{boolean-operations-with-r}}

\begin{Shaded}
\begin{Highlighting}[]
\FunctionTok{which}\NormalTok{(v }\SpecialCharTok{==} \FunctionTok{max}\NormalTok{(v))}
\end{Highlighting}
\end{Shaded}

\begin{verbatim}
[1] 5
\end{verbatim}

\begin{Shaded}
\begin{Highlighting}[]
\FunctionTok{which}\NormalTok{(v }\SpecialCharTok{==} \FunctionTok{min}\NormalTok{(v))}
\end{Highlighting}
\end{Shaded}

\begin{verbatim}
[1] 1
\end{verbatim}

\begin{Shaded}
\begin{Highlighting}[]
\NormalTok{s }\OtherTok{\textless{}{-}} \DecValTok{1}\SpecialCharTok{*}\NormalTok{t}
\NormalTok{s}
\end{Highlighting}
\end{Shaded}

\begin{verbatim}
[1] 0 0 0 0 1 1
\end{verbatim}

\begin{Shaded}
\begin{Highlighting}[]
\NormalTok{v }\OtherTok{\textless{}{-}} \FunctionTok{c}\NormalTok{(}\DecValTok{10}\NormalTok{,}\DecValTok{20}\NormalTok{,}\DecValTok{70}\NormalTok{,}\DecValTok{30}\NormalTok{,}\DecValTok{60}\NormalTok{,}\DecValTok{50}\NormalTok{)}
\FunctionTok{all}\NormalTok{(v }\SpecialCharTok{\textgreater{}} \DecValTok{5}\NormalTok{)}
\end{Highlighting}
\end{Shaded}

\begin{verbatim}
[1] TRUE
\end{verbatim}

\begin{Shaded}
\begin{Highlighting}[]
\FunctionTok{any}\NormalTok{(v }\SpecialCharTok{\textless{}} \DecValTok{5}\NormalTok{)}
\end{Highlighting}
\end{Shaded}

\begin{verbatim}
[1] FALSE
\end{verbatim}

\end{column}

\begin{column}{0.04\textwidth}

~

\end{column}

\begin{column}{0.48\textwidth}

\hypertarget{boolean-operations-with-python-1}{%
\subsubsection{Boolean operations with
Python}\label{boolean-operations-with-python-1}}

\begin{Shaded}
\begin{Highlighting}[]
\NormalTok{np.where(v }\OperatorTok{==} \BuiltInTok{max}\NormalTok{(v))}
\end{Highlighting}
\end{Shaded}

\begin{verbatim}
(array([4]),)
\end{verbatim}

\begin{Shaded}
\begin{Highlighting}[]
\NormalTok{np.where(v }\OperatorTok{==} \BuiltInTok{min}\NormalTok{(v))}
\end{Highlighting}
\end{Shaded}

\begin{verbatim}
(array([0]),)
\end{verbatim}

\begin{Shaded}
\begin{Highlighting}[]
\NormalTok{s }\OperatorTok{=} \DecValTok{1}\OperatorTok{*}\NormalTok{t}
\NormalTok{s}
\end{Highlighting}
\end{Shaded}

\begin{verbatim}
array([[0, 0, 0, 0, 1, 1]])
\end{verbatim}

\begin{Shaded}
\begin{Highlighting}[]
\NormalTok{v }\OperatorTok{=}\NormalTok{ np.array([}\DecValTok{10}\NormalTok{,}\DecValTok{20}\NormalTok{,}\DecValTok{70}\NormalTok{,}\DecValTok{30}\NormalTok{,}\DecValTok{60}\NormalTok{,}\DecValTok{50}\NormalTok{])}
\BuiltInTok{all}\NormalTok{(v }\OperatorTok{\textgreater{}} \DecValTok{5}\NormalTok{)}
\end{Highlighting}
\end{Shaded}

\begin{verbatim}
True
\end{verbatim}

\begin{Shaded}
\begin{Highlighting}[]
\BuiltInTok{any}\NormalTok{(v }\OperatorTok{\textless{}} \DecValTok{5}\NormalTok{)}
\end{Highlighting}
\end{Shaded}

\begin{verbatim}
False
\end{verbatim}

\end{column}

\newline

\hypertarget{conclusion}{%
\subsection{Conclusion}\label{conclusion}}

This conclude the document for now, I'll try to update it if I find
useful tips. Hope this will help you in the future.

\end{document}
